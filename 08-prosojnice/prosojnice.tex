\documentclass{beamer}
\title{Matematični izrazi in uporaba paketa \texttt{beamer}}
\subtitle{\emph{Matematičnih} nalog ni treba reševati!}
\institute{Fakulteta za matematiko in fiziko}
\date{}

\usepackage{predavanja}

{\theoremstyle{plain}
\newtheorem{izrek}{Izrek}[section]
\newtheorem{posledica}[izrek]{Posledica}
}

{\theoremstyle{definition}
\newtheorem{definicija}[izrek]{Definicija}
\newtheorem{vaja}[izrek]{Vaja}
}


\begin{document}
\frame{\titlepage}
\begin{frame}
    \frametitle{Kratek pregled}
    \tableofcontents%[pausesections]
    
\end{frame}

\section{Paket \texttt{beamer}}
\input{prosojnice/1-paket-beamer.tex}

\section{Paketa \texttt{amsmath} in \texttt{amsfonts}}
\input{prosojnice/2-paketa-amsmath-amsfonts.tex}

\section[Matematika, 1. del\\\large{Analiza, logika, množice}]{Matematika, 1. del}
\input{prosojnice/3-analiza-logika-mnozice.tex}

\section{Stolpci in slike}

\section{Paket \texttt{beamer} in tabele}

\section[Matematika, 2. del\\\large{Zaporedja, algebra, grupe}]{Matematika, 2. del}

\end{document}